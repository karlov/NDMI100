\section{\texorpdfstring{Perfect security, secret sharing, symmetric ciphers}{Perfect security, secret sharing, symmetric ciphers}}
\vspace{5mm}
\large

\subsection{Perfect Security}
%2nd lecture, but belongs to 3rd by topic
\begin{definition}
	\textbf{One time pad (also known as Vernam's Cipher)} - is the only! cipher, we can prove secure.

	The key $k \in \{ 0, 1 \}^n $ is chosen uniformly at random. The message $x \in \{ 0, 1 \}^n$.

	cipher text is defined as
	\[ y = x \oplus k \iff y_i = x_i \oplus k_i \]

	Decryption, using the associativity of XOR.
	\[ y \oplus k = (x \oplus k) \oplus k = x \oplus (k \oplus k) = x \]
	Therefore encryption function is the same as decryption.

	From the observation \cref{random_bit}, XOR with random bits produces uniform random bit.
	Only the size of the information could be deduced from the cipher text. Extremely strong result!
	No matter how much computational power the attacker has, he cannot deduce plain text from cypher text.

	For other ciphers we can only prove that brute force attack is extremely slow, or $P = NP$ should be true to break the cipher.

	Such property is called \textbf{Informational-theoretic security} or \textbf{perfect security}.

\end{definition}

\begin{definition}\label{gen_one_time_p}
	\textbf{Generilized one-time pad}

	Consider some group additive G $(G, +, 0, -)$.

	We pick some $k \in G$ randomly, then
	\[ E(x, k) = x + k, D(y, k) = y - k = y + (k)^{-1} \]

	For example we can choose $\Z_t$.
\end{definition}

\begin{definition}
	A symmetric cipher is \emph{perfectly secure} iff
	\[ \forall x \forall y Pr_k[ E(x, k) = y ] = Pr_k[ D(y, k) = x ] = const \]

	For any plain text, all cipher texts should be equally likely.
\end{definition}

\begin{example}
	For example, for generalized one time pad \cref{gen_one_time_p}
	\[ \forall x \forall y \exists !k: x + k = y \Rightarrow k = y + x^{-1} \]

	Therefore the probability is $\frac{1}{|G|}$. As a result, one time pad is perfectly secure.
\end{example}

\begin{properties}
Usefulness of the OTP:
	\begin{enumerate}
		\item never use same key twice. Since xor of the cypher text is the same as xor of the plain text.
		\[ x_1 \oplus k = y_1 \land x_2 \oplus k = y_2 \Rightarrow y_1 \oplus y_2 = x_1 \oplus k \oplus x_2 \oplus k = x_2 \oplus x_1 \]
		\item Codebook - a long list of generated Keys.
			It has already happen during WW2, soviet agents ran out of keys in codebook and started to used same keys from the beginning.
			Not only were Americans able to read new messages, they have also decoded old ones.
		\item replace randomness by pseudorandomness. A key is produced by PRNG (pseudo random number generator) which is parametrized by another key and ideally by nonce (initialization vector).
		\item if attacker changes specific bit of the cipher text same plain text bit is changed. Since XOR is done bit by bit.
	\end{enumerate}
\end{properties}

\begin{theorem}[Perfect cipher]
	If \# of keys $<$ \# of messages $\Rightarrow$ the cipher is not perfectly secure.
\end{theorem}
\begin{proof}
	Let's take some cipher text $y$ and get all plain text that a mapped by the inverse of the encryption function to the some $T \subset $ Plain Texts. Since \# of keys $<$ \# of messages $\Rightarrow$ some of the keys $\notin T$.
	Therefore
	\[ Pr_k[x \notin T \land D(y, k) = x] = 0, Pr_k[x \in T \land D(y, k) = x] > 0 \]
	And the distribution of keys are not uniform. Cipher is not perfectly secure by the definition.
\end{proof}

\begin{note}
	Perfect security is also called \textbf{Shannon security}. He is the father of the information theory and cryptography.
\end{note}

\subsection{Secret Sharing}

Assume we have a secret information. We want to break secret into shared. The whole secret can be restored only having all the pieces.
We can use the OTP:

$S_1, ..., S_{k-1}$ are random numbers. $S_k = x \oplus S_1 \oplus ... \oplus S_{k-1}$.
Shareholders can decode x by xoring all $S_i$. Otherwise they will get random numbers.

\begin{definition}
	$(k, l)$ - threshold scheme. Split secret $x$ into $k$ shares $S_1, ..., S_k$.
	Having any $l$ shared agents can decode $x$. However $X$ cannot be decoded with $< l$ \# of shares. No info should be given.
\end{definition}

\begin{example}
	$(k, 2)$ scheme. Let F be finite field, the secret is a function $f = ax + b$ s.t.:
	\[ f(0) = x, f(1) \in_R F \]

	Geometrically, we have $x$ as the point with coordinates $(0, t)$ for some $t$.
	All $S_i$ for a straight line. As line can be deduced by any 2 points, 2 agents can decode $x$ using their points $S_i$.

	Distribution of the secrets $x$ is uniform $\Rightarrow$ perfectly secure.
\end{example}

Let's consider polynomials over field F.
According to results from Algebra, a polynomial of degree $d$ has at most $d$ roots.

Graph of the polynomial is a set of values for every point $\in F$.
Polynomials with the same graph share the roots $\Rightarrow$ should be equal.

\begin{theorem}[Lagrange interpolation]
	For every $x_1, ..., x_d, x_i \neq x_j, i \ne j, P(i) = 0$ distinct roots of P and $y_1, ..., y_d$
	\[ \exists! P, deg(P) < d: \forall i\ P(x_i) = y_i \]
\end{theorem}
\begin{proof}
	Let $ \overline{x} = (x_1, ..., x_d) \in F^d$ be a vector of coefficients of polynomial P, X is a set of all such vectors. $|X| = |F|^d$.

	Let Y be a set of all choices of $y_i$, graphs of the polynomials.

	Finally, let $f:X \to Y$ be a map from polynomials to graphs of polynomials. Which evaluates polynomial at d points.

	Since polynomials with the same graph are equal, $f$ is injective. $|X| = |Y| \Rightarrow f$ is bijection.
\end{proof}

\begin{consequence}
	Scheme based on graphs of the polynomials is perfectly secure, since according to the Lagrange theorem we have bijection between them $\Rightarrow$ uniform distribution.
\end{consequence}

\begin{example}
	$(k, l)$ scheme. Instead of lines we pick some polynomial P, $deg(P) = d$. Then we randomly pick $k$ values of the P.
	Having exactly $k$ shared agents can obtain P. Otherwise, all $X = P(0)$ are uniformly distributed.
\end{example}
