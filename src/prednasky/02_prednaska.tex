\section{\texorpdfstring{Auction protocol, attacks, Security level}{Auction protocol, attacks, Security level}}
\vspace{5mm}
\large
\begin{observation}\label{random_bit}
	Any fixed bit XOR random bit (uncorrelated with fixed bit) produces random bit.
\end{observation}

\begin{example}
	\textbf{Bit commitment protocol}

	Toss a coin over a phone call. We assume, that 2 parties communicate over some line.
	They want to generate a random bit, however they do not trust each other.

	If Alice sends her random bit to Bob, he send his random bit back, then XOR should produce random bit. However, Bob could cheat, and choose bit according to his need.

	If Alice generates a random bit, then receives Bob's random bit, then she could cheat and change her bit in the process. Similarly for Bob.

	So, we need better approach. Alice generates random word of fixed length, let's say 128 bit. $x \in \{ 0, 1 \} ^{128}$. Alice sends $h(x)$. Bob sends his bit $b$.

	Then Alice send first bit of X as her random bit. Both do XOR the bits.
	The communication protocol ensures that Bob can verify Alice's random string by computing the hash.

	X should be long, otherwise Bob could brute force to check all possible hashes.
\end{example}

\begin{observation}
	In many communication protocols hash functions and RNG are more important than ciphers.
\end{observation}

\begin{example}
	Alice wants to participate in the auction. However, she could not do so in person. She chooses Bob as her representative.

	An auction is public and it has a broadcast which cannot be changed by active attacker. Since public will notice.

	Alice send 2 messages: Stop, Add <money>.

	In every protocol design we should consider:
	\begin{enumerate}
		\item Against whom we are defending
		\item For how long information should be secret.
	\end{enumerate}

	In the auction example, information should be secret for a couple of minutes maximum.
	However, Alices actions should be kept secret before Bob acts on the auction. Otherwise other participants could do a perfect counter move.

	A symmetric cipher will be used. We assume, that Alice and Bob met in person before the auction and exchanged the key.

	Problems of the Protocol and the possible solutions:
\begin{enumerate}

	\item If the messages use the same encryption key several times or, length of messages is the same, Eve could distinguish Stop from Add.
	Therefore, an \textbf{nonce} should be added to the message in order to make them the same length and prevent attacker from guessing using the length.

	\item The difference in length between Stop and Add. To fix this, we add padding to every message making all messages the same fixed length.

	\item Eve could sniff the message and check what were Bob's bid on the auction. Later she could use this message pretending to be Alice. Usually called \textbf{Replay attack}.
	In order to protect the communication from replay attack, Alice will add a sequence number of the current message (monotonically increasing sequence).
	If Bob receives the same sequence number twice, he considers it as an attack.

	\item In practice, Eve could disable communication line for some time and resend all messages later. Timestamp could be a solution, however it requires a sync clock between parties. Which is hard to do secretly.

	An alternative solution would be to add NOP (no operation) message and send messages periodically.

	\item Mallory can randomly change bit in the Cipher text. For many good ciphers, altering one bit in the cipher text will result in 1 bit change in plain text.
		To prevent that Alice should sign the message using \textbf{Message Authentication Code (MAC)}. The basic example of MAC creates $h(k_s | E_k(x))$, where $k_s$ is a new symmetric key.

		As a result, hash of the message no longer match if Mallory changed some bits.
	\item If Alice and Bob use same keys for the 2nd time, Eve could replay messages from the previous session. To prevent that they can use a different pair of keys every time.
		Alternatively, a \textbf{session Id} could be added to the message. Should be a secret, known only to Alice and Bob. We can also choose session id as nonce and make it a parameter of MAC.

	\end{enumerate}
\end{example}

\begin{observation}
	Every part of the protocol could be attacked. Protocol is as strong as the weakest part.
\end{observation}

\begin{definition}
	Authenticity - message was not changed during transfer.

	Secrecy - no one decoded the plain text of the message.

	In almost all cases, we want to combine both properties.
\end{definition}

\begin{observation}
	Changing the encryption key for every message is generally no a good idea for several reasons.
	First of all, an attacker has much smaller set of keys to check (entropy reduction).
	Secondly, many RNGs require some hear up. For example, an RNG has to precompute some values.

	We can however, derive key from single Master key. Later we will prove that adding \textbf{nonce} is as strong as changing the key.
	Frequent change of keys makes protocol harder to analyze.
\end{observation}

\subsection{Types of attacks}

\begin{example}
	Known cipher text - an attacker knows cipher text and wants to recover the plain text from it.
\end{example}

\begin{example}
	Known plain text - an attacker knows cipher text and plain text. He wants to recover the Key from them. Nonce are still not known, so only part of the plain text is known.
\end{example}

\begin{example}
	Chosen plain text - an attacker chooses plain text and gives it to the victim for encryption. Goal is again to recover the key.
\end{example}

\begin{example}
	Distinguishing attack - e.g. distinguish Stop and Add commands in the Auction protocol (by the length or other property of the cypher text).
\end{example}

How to measure strength of the cryptographic primitives. Quantification. For that we define security level (in bits).

\begin{definition}
	Security level - is the number of operations required to break some cryptographic primitive.
	Ideally we want security level to be equal to the key length. Meaning that the attacker has to check all possible keys in order to break the primitive.

	Security level of the good cryptographic primitive should be beyond computational power of any computer on the Earth.
\end{definition}

\begin{example}
	\textbf{Birthday attack} - the security level of the primitive is only half of the expected.

	For example, let's consider Challenge-response auth scheme.
	The nonce used in the protocol has $b$ bits. One could expect the security level to be $2^b$.
	Which is not true, because after $2^{b/2}$ the collision is likely to occur.

	This attack is related to the so called \textbf{Birthday paradox}.
	If there is 23 people in the room, the probability of the same birthday is actually $ \geq 1/2$.
\end{example}

\begin{theorem}[Birthday attack probability]
	The probability of the repetitions in nonces is $e^{-\frac{n^2}{2m}}$.
\end{theorem}
\begin{proof}
	Let's compute the probability that function $f:[n] \to [m]$ that generates nonces is injective. Where $n$ is number of runs, and $m$ is number of nonces.
	Then probability of A is \# of injective functions over all functions
	\[ Pr_f[A] = \frac{m(m-1)(m-2)...(m-n+1)}{m^n}\]
	Because we choose 1 nonce every run, number of available nonces decreases by 1 every run. Then we rewrite the sum as
	\[ \frac{m}{m} \frac{m-1}{m} ... \frac{m-n+1}{m} = 1 (1 - \frac{1}{m}) (1 - \frac{2}{m}) ...\]
	According to Calculus, for small x: $1 - x \approx e^{-x}$.
	\[ Pr_f[A] \approx 1 e^{-\frac{1}{m}} e^{-\frac{2}{m}} ... = e^{-\frac{1 + 2 + ... + n - 1}{m}} = e^{-\frac{n(n-1)}{2m}} \leq e^{-\frac{n^2}{2m}} \]

	For what values of $n$ and $m$ the probability is $\frac{1}{2}$?
	\[ e^{-\frac{n^2}{2m}} = 1/2 \Rightarrow \frac{n^2}{2m} = -2 \ln\left(\frac{1}{2}\right) \approx 1.39 \Rightarrow n^2 \approx m \]
\end{proof}

\begin{consequence}
	Number of runs required to reach $\frac{1}{2}$ probability of collisions is $\sqrt{m}$. Therefore sequrity level of Challenge-response auth scheme is $2^{b/2}$.
\end{consequence}


\begin{example}
	Assume Alice wants to send to Bob new private key, knowing his public asymmetric key $k_{pub}^B$. To do so, she encodes new key with Bobs key
	\[ E_{asym}(k_{new}, k_{pub}^B) \]

	Eve has access to $k_{pub}^B$, so she can guess a key $k_{g}$ and compare the results. By the assumption, $k_{new}$ is random, so guessing 1 key will not work.
	More sophisticated approach is to precompute a huge set of Keys.
	After some time, encrypted Key sent by Alice will be on of the precomputed one by Eve.
	As a result, Eve would have both encrypted and plain text of the Key.
\end{example}

\begin{theorem}[Success of precomputed table]
	Eve requires $tn \approx m$ (where $t$ is size of the precomputed table, $n$ is number of runs, $m$ is size of Keys set) to guess the key selected by Alice.
	In other words, key selected by Alice will be in the precomputed subset by Eve.
\end{theorem}
\begin{proof}
	Let's compute probability that random function $f$ generates a key that would not be in the precomputed subset.
	\[ P_f[f\ avoids \ subset] = (1 - \frac{t}{m})^n \]
	Since probability of $f$ hitting the subset is $\frac{t}{m}$ and key selection are independent.
	By the same approximation as in previous theorem
	\[ P_f[f\ avoids \ subset] = e^{-\frac{-tn}{m}} \Rightarrow tn \approx m \]

	For example $t = n = \sqrt{m}$. Again security level is $\leq$ 1/2 \# of bits.
\end{proof}
